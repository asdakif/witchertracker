\documentclass[12pt]{article}
\usepackage{geometry}
\geometry{a4paper, margin=1in}
\usepackage{graphicx}
\usepackage{listings}
\usepackage{color}
\usepackage{titlesec}
\usepackage{enumitem}
\usepackage{hyperref}
\usepackage{amsmath}

\title{Witcher Tracker\\CMPE 230 – Project 3 Documentation}
\author{Akif Yıldırım \\ 2020400144}
\date{\today}

\begin{document}

\maketitle

\tableofcontents

\newpage

\section{Project Overview}
The Witcher Tracker is an inventory-event management system that simulates Geralt's interactions in the Witcher universe. The system is implemented in C++ using Object-Oriented Programming and handles commands related to looting, brewing potions, learning monster weaknesses, and querying Geralt’s inventory, bestiary, and alchemy formulas.

\section{Design and Implementation}

\subsection{Class and Object Design}
\begin{itemize}
    \item \textbf{Inventory}: Manages ingredients, potions, and trophies.
    \item \textbf{Bestiary}: Stores knowledge about monsters and their effective counters.
    \item \textbf{Alchemy}: Maintains potion recipes and handles brewing logic.
\end{itemize}

\subsection{Execution Flow}
\begin{enumerate}
    \item The program reads a line of input prefixed with \texttt{>>}.
    \item The line is passed to the parser, which identifies the command type.
    \item The appropriate handler is called, and the result is printed.
\end{enumerate}

\section{Challenges and Solutions}
\begin{itemize}
    \item \textbf{Flexible input parsing}: Regex patterns were carefully constructed to allow varying whitespace but disallow invalid characters or structure.
    \item \textbf{Input sanitization}: We had to trim and normalize input for accurate parsing and validation.
\end{itemize}

\section{Usage Guide}

\subsection{Compilation}
To compile the program:
\begin{verbatim}
make
\end{verbatim}

This will generate an executable named \texttt{witchertracker}.

\subsection{Execution}
To run the program:
\begin{verbatim}
./witchertracker
\end{verbatim}

\subsection{Example Inputs and Outputs}
\begin{verbatim}
>> Geralt loots 4 Vitriol, 3 Rebis
Alchemy ingredients obtained

>> Geralt learns Igni sign is effective against Harpy
New bestiary entry added: Harpy

>> Geralt encounters a Harpy
Geralt defeats Harpy

>> Total ingredient ?
3 Rebis, 4 Vitriol

>> What is in Black Blood ?
3 Vitriol, 2 Rebis, 1 Quebrith
\end{verbatim}

\section{Code Structure Summary}

\begin{itemize}
    \item \texttt{main.cpp}: Entry point
    \item \texttt{WitcherTracker.h/.cpp}: Main class that holds inventory, bestiary, and alchemy objects.
    \item \texttt{Inventory.h/.cpp}: Manages adding/removing/querying ingredients, potions, and trophies.
    \item \texttt{Bestiary.h/.cpp}: Manages effective signs and potions against monsters.
    \item \texttt{Alchemy.h/.cpp}: Manages potion recipes and brewing logic.
\end{itemize}

\section{AI Assistant Usage}
ChatGPT was used for:
\begin{itemize}
    \item Suggesting regex patterns for flexible yet strict grammar validation.
    \item Debugging segmentation faults in input parsing stages.
\end{itemize}
\end{document}
